\section{数列极限与实数 Limits of Sequences \& Real
Numbers}\label{ux6570ux5217ux6781ux9650ux4e0eux5b9eux6570-limits-of-sequences-real-numbers}

\subsection{数列极限的定义}\label{ux6570ux5217ux6781ux9650ux7684ux5b9aux4e49}

对于数列\(\{x_n\}_{n=1}^{\infty}\),如果存在\(l\in\mathbb{R}\),使得对于任意\(\varepsilon>0\),总能找到一个对应的\(N\),使得对于任意满足\(n>N\)的\(x_n\),都有\(\vert x_n-l\vert <\varepsilon\),则称数列\(\{x_n\}_{n=1}^{\infty}\)\textbf{收敛}(到\(l\)),或者说数列\(\{x_n\}_{n=1}^{\infty}\)趋于\(l\),或称\(l\)是数列\(\{x_n\}_{n=1}^{\infty}\)的极限,记为

\[
\lim_{n\rightarrow\infty}x_n = l\quad\text{或}\quad x_n\rightarrow l,\quad n\rightarrow\infty
\]

对于数列而言,在研究极限时仅考虑\(n\rightarrow\infty\)的情形,这一点和后面要讲到的函数的极限是不同的。因此我们可以更精简地记为\(\displaystyle \lim x_n = l\)和\(x_n\rightarrow l\).

\begin{quote}
\begin{itemize}
\tightlist
\item
  直白地说,上面的定义是在描述:通过选择合适的\(N\),我们可以将\(\vert x_n - l\vert\)控制到\textbf{任意小}.
  即无论给定一个多么小的正数\(\varepsilon\),总能找到若干合适的\(N_\varepsilon\),从而将数列在超过\(N_\varepsilon\)的部分严格地控制在区间\((l-\varepsilon, l+\varepsilon)\)以内.
\item
  我们用\(N_\varepsilon\)而不是\(N(\varepsilon)\),来澄清\(N\)和\(\varepsilon\)并不是一个严格的映射(函数)关系。对于每一个\(\varepsilon\),我们有不止一种选择\(N\)的方式。假如我们找到了一个合适的\(N^\star_\varepsilon\),则显然\(N^\star_\varepsilon+1\)、\(N^\star_\varepsilon+4\)甚至\(N^\star_\varepsilon+10000\)等也可以作为\(N_\varepsilon\)的选取方式。
\end{itemize}
\end{quote}

对于数列\(\{x_n\}_{n=1}^{\infty}\),只要能找到符合上述定义的\(l\in\mathbb{R}\),则称其为\textbf{收敛}(convergent)的,否则是\textbf{发散}(divergent)的。

\subsection{无穷小量和无穷大量}\label{ux65e0ux7a77ux5c0fux91cfux548cux65e0ux7a77ux5927ux91cf}

\subsubsection{无穷小量和无穷大量的概念}\label{ux65e0ux7a77ux5c0fux91cfux548cux65e0ux7a77ux5927ux91cfux7684ux6982ux5ff5}

如果数列\(\{x_n\}_{n=1}^{\infty}\)的极限是0,则其又称为\textbf{无穷小量},记作
\[
x_n = o(1),\quad n\rightarrow\infty
\]

因此,数列\(\{x_n\}_{n=1}^{\infty}\)收敛至\(l\)这一命题\textbf{等价于}:\(x_n-l\)是无穷小量。

\begin{quote}
不能把无穷小量视作数字0(虽然有些场景下这样处理确实很爽,但有些场景则会出现0/0或者\(0\times\infty\)的复杂情形)。无穷小量并不是一个静态的数字,而是一个动态的变量,一个被下标\(n\)调控的量。我们可以调控\(n\)以使得这个变量的值可以\textbf{任意小},因而称为无穷小量。
\end{quote}

如果数列\(\{x_n\}\)满足:对任意的\(E>0\),总存在\(N_E>0\),使得对于任意\(n>N_E\),总有\(\vert x_n\vert > E\),则称数列\(\{x_n\}\)为\textbf{无穷大量}。记为\(\lim x_n=\infty\)或者\(x_n\rightarrow\infty\).
特别地,如果\(x_n\)在这一过程中保持符号,例如,对任意的\(E>0\),总存在\(N_E>0\),使得对于任意\(n
> N_E \),总有 \(x_n > E\)
(\textbf{注意这里没有绝对值号}),则记为\(\lim x_n=+\infty\)或\(x_n\rightarrow +\infty\)。如果\(\{x_n\}\)满足\(-x_n\rightarrow+\infty\),则可以记作\(\lim x_n=-\infty\)或\(x_n\rightarrow -\infty\).

\begin{quote}
\begin{itemize}
\tightlist
\item
  上面的定义是在描述:通过选择合适的\(N\),我们可以将\(\vert x_n\vert\)(或者\(x_n\),或者\(-x_n\))控制到\textbf{任意大}.
\item
  无穷大量同样也不是一个静态的数,而是一个变量。
\end{itemize}
\end{quote}

\subsubsection{无穷小量和无穷大量的性质}\label{ux65e0ux7a77ux5c0fux91cfux548cux65e0ux7a77ux5927ux91cfux7684ux6027ux8d28}

(1)
\textbf{加法}:\textbf{有限个}无穷小量的和仍然是无穷小量,即\(o(1)+o(1)=o(1)\).

(2)
\textbf{乘法}:若\(\{x_n\}\)有界,则\(\{x_n\}\)和一个无穷小量的积也是无穷小量。

(3)
\textbf{倒数}:若\(\{x_n\}\)是无穷大量,则\(\displaystyle \left\{\frac{1}{x_n}\right\}\)是无穷小量;若\(\{x_n\}\)是无穷小量且\(x_n\neq 0\),则\(\displaystyle \left\{\frac{1}{x_n}\right\}\)是无穷大量.

\subsection{极限的相关性质 Properties of
Limits}\label{ux6781ux9650ux7684ux76f8ux5173ux6027ux8d28-properties-of-limits}

\subsubsection{单个数列极限的性质}\label{ux5355ux4e2aux6570ux5217ux6781ux9650ux7684ux6027ux8d28}

(1)
\textbf{保序性、保号性}:假定\(\lim x_n = a\)且\(a > p\),则从某一项开始(或者说,存在一个\(N_p\),使得对于任意的\(n>N_p\)),总有\(x_n>p\)。同理,若\(a < q\),则从某一项开始,总有\(x_n < q\)。是为极限的\textbf{保序性}。

\begin{quote}
证明思路是很直接的:以\(a>p\)为例,只需要取一个足够小的\(\varepsilon\)满足\(a-\varepsilon>p\)(例如可以取\(\displaystyle \varepsilon=\frac{a-p}{2}\))。则对任意\(n>N_\varepsilon\),总有\(x_n>a-\varepsilon>p\)。另一种情况同理。
\end{quote}

特别地,当\(a>0\)时,可以取\(p=0\);当\(a<0\)时,可以取\(q=0\)。即:若数列趋于一个正数,则从某一项开始数列恒正;若数列趋于一个负数,则从某一项开始数列恒负。是为数列的\textbf{保号性}。

(2)
\textbf{有界性}:收敛数列必有界。即若\(\{x_n\}\)收敛到\(a\),则存在\(M>0\),使得对于\textbf{所有}\(x_n\),均有\(\vert x_n\vert \leqslant M\).

\begin{quote}
任意取\(\varepsilon\)和一个对应的\(N_\varepsilon\),则对于\(n>N_\varepsilon\)的部分,有\(\vert x_n \vert < \vert a\vert + \varepsilon\);对于\(1\leqslant n \leqslant N_\varepsilon\)的部分,这\(N_\varepsilon\)个有限的\(\vert x_n \vert\)中总能照到一个最大值,令\(\displaystyle M=\max\left\{\max_{1\leqslant n\leqslant N_\varepsilon}|x_n|, \vert a\vert + \varepsilon\right\}\)即可。
\end{quote}

(3)
\textbf{唯一性}:收敛数列有且仅有一个极限。即若\(x_n\rightarrow a\)且同时\(x_n\rightarrow b\),则必然有\(a=b\)。

\subsubsection{多个数列极限的性质}\label{ux591aux4e2aux6570ux5217ux6781ux9650ux7684ux6027ux8d28}
