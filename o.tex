\section{随机过程的基本概念和分类 Fundamental concepts \&
classification}\label{ux968fux673aux8fc7ux7a0bux7684ux57faux672cux6982ux5ff5ux548cux5206ux7c7b-fundamental-concepts-classification}

\(\newcommand{\bs}[1]{\boldsymbol{#1}}\)

\subsection{随机过程简介}\label{ux968fux673aux8fc7ux7a0bux7b80ux4ecb}

在概率论中,我们对单个\textbf{随机变量}(random variables,
r.v.)进行了研究。进一步地,由参数\(t\)作为``索引''的一组随机变量则称为\textbf{随机过程}(stochastic
processes),记为\(\bs{X}=\{X(t):t\in T\}\).
这里\(T\)是指标集,对于指标集中的每一个\(t\),对应的\(X(t)\)都是一个单独的\emph{随机变量}。因此,随机过程就是往随机变量中引入了``空间''的概念。在大部分场合中,指标\(t\)经常代指``时间''。

随机过程有多种产生或描述方式。例如,一种常见的描述方式是\(X(t)\sim \mathcal{P}\left(\bs{\theta}(t)\right)\),其中\(\mathcal{P}\)是某个分布族,而\(\bs{\theta}\)是该分布族的参数。这表明在任意时间点\(t\),随机变量\(X(t)\)服从分布\(\mathcal{P}\),且该分布的参数与\(t\)有关。

\begin{quote}
例如,随机过程\(X(t)\sim\mathcal{N}(t,1)\),表明每一点\(X(t)\)都服从正态分布,且均值就是\(t\)。
\end{quote}

还有一种方式是基于对某个\emph{确定函数}的改造:给定一个确定的函数\(f:T\rightarrow\mathbb{R}\),和一个随机变量\(\xi\sim\mathcal{P}(\bs{\theta})\),定义\(X(t)=f(h(t,\,\xi))\),则这也是一个随机过程。

\begin{quote}
常见的例子就是随机相移:例如考虑\(f(t)\)是一个确定函数,而\(u\sim\mathcal{U}[0,T_0]\),则\(f(t+u)\)向原来的函数\(f\)引入了一个随机相移,这便是一个随机过程.
\end{quote}

\subsection{随机过程的数字特征}\label{ux968fux673aux8fc7ux7a0bux7684ux6570ux5b57ux7279ux5f81}

\subsection{随机过程分类}\label{ux968fux673aux8fc7ux7a0bux5206ux7c7b}
